\documentclass[conference, letterpaper]{IEEEtran}
%\documentclass[a4paper]{article}

\ifCLASSINFOpdf \else \fi

\usepackage[pass]{geometry} \usepackage{tabularx} \usepackage{graphicx}
\usepackage{amssymb} \usepackage{algorithm} \usepackage{algorithmic}
\usepackage{amsmath} \usepackage{booktabs} \usepackage{caption}

\usepackage{cite} \usepackage[numbers,sort&compress]{natbib}
\captionsetup[table]{position=top,labelfont={sc},textfont={sl}}

\renewcommand{\thetable}{\Roman {table}}

\title{ Implementation and Evaluation of Asymmetric link-based AODV
Routing Protocol for Low Power and Lossy Networks (AODV-RPL)}

\author{ \IEEEauthorblockN{Lavanya HM, Anand S V R, Malati Hegde} \
\IEEEauthorblockA{ECE Department, Indian Institute of Science \\ \{lavanyahm,
anandsvr, malati\}@iisc.ac.in} }

\begin{document}
\maketitle
\begin{abstract}

Reactive route discovery for Point-to-Point (P2P) traffic flows is crucial in
Low power and Lossy Networks (LLNs) for several use cases including home and
building automation. The Routing Protocol over LLNs (RPL) standardized by IETF
cannot be efficiently used for P2P flows that result in network congestion at
the DODAG (Destination Oriented Directed Acyclic Graph) root, and packet delays.
While P2P-RPL protocol is designed to address this problem, both these routing
protocols does not consider asymmetric wireless links in their route
computation.

The AODV-RPL protocol draft which has been adopted by the roll working group of
IETF is a reactive P2P route discovery mechanism for hop-by-hop routing based on
Ad Hoc On-demand Distance Vector Routing (AODV) works in conjunction with RPL
operating in the storing mode. In this paper we evaluate the performance of
AODV-RPL on Cooja simulator under different network topologies and node
densities for symmetric and asymmetirc links. Our extensive simulations show
that AODV-RPL provides better Packet Delivery Ratio (PDR) and delay performance
for several source and destination pairs compared to traditional RPL. Our
simulation experiments also indicate that while AODV-RPL P2P routes are better
compared to traditional RPL, there are several node-pairs where the relative hop
distance difference is close. Based on this important observation, we suggest a
simple hybrid routing mechanism where RPL routes are re-used to reduce routing
protocol overheads due to AODV-RPL.



\end{abstract}

Keywords—AODV, DODAG, RPL and P2P links.  \section{Introduction}

Internet of Things (IoT) consists of thousands of constrained devices that
include sensors and actuators, interact with their environment and are
inter-networked together, and become accessible through the internet
\cite{iot,industry,iotservey}.  Currently 6LoWPAN \cite{iotstack} is the
proposed standard to translate the IPv6 packets to the IEEE 802.15.4 MAC frame
through compression and fragmentation. The design of a robust routing protocol
plays a pivotal role to enhance the performance of the memory and power
constrained nodes in a dynamically changing network \cite{lowpan}. To achieve
this, IETF proposed a proactive based IPv6 distance vector routing protocol for
Low-power and Lossy Networks, named RPL \cite{RFC6550}.

The RPL is primarily designed to build and maintain loop-free routing paths over
mutli-hop networks to cater to the requirements of many IoT real-world
applications which are typically multi-point-to-point, or point-to-multipoint in
nature. The assumptions that are made in the RPL have however been found to be
limiting when applied to networks for home and building automation applications
characterized by message reliability and latency requirements between
communicating nodes that belong to the same network, in other words, P2P
communication. For instance, a remote control operation for controlling home
appliances. The reasons for this shortcoming is best explained by a short
description of RPL with respect to P2P communication.

Constrained nodes within the LLN that are running RPL protocol are connected by
construction of a Destination Oriented Directed Acyclic Graph (DODAG). The LLN
border router, also referred to as the DODAG root, initiates the DIO control
messages to form a DODAG for a specific instance requirement, termed Objective
Function. With this, a node within the LLN is able to select its parent node,
which is often the best candidate amongst a set of next-hop neighbours. A path
is established from the nodes in the network through the DODAG root. For
accommodating nodes of differing resource capabilities, RPL offers two modes of
operation. In case of non-storing mode which is employed in situations where the
nodes are constrained by processing and memory capabilities, all the routes are
stored in the DODAG root. Because of this, every P2P communication takes place
via the DODAG root. It results in longer routes between peers, and radio
congestion at the root. On the other hand, in the storing mode of operation,
where the nodes are resourceful, every node in the network maintains downward
routes its child nodes. Here too, the routes are sub-optimal because the packet
from the originating node ends up traversing upwards till it reaches a common
ancestor that holds the downward path to the destination node.

To avoid such sub-optimal routes, a reactive Point to Point \cite{RFC6997}
protocol(P2P-RPL) has been proposed. Although P2P-RPL is more efficient for P2P
traffic in non-storing mode, its address vector usage in storing mode causes
large network overhead.  Also, both traditional RPL and P2P-RPL does not
consider the presence asymmetric links for the route computation.

AODV-RPL \cite{I-D.ietf-roll-aodv-rpl}, is a reactive protocol that has been
proposed for both symmetrical and asymmetrical links, and to overcome the
disadvantages of RPL and P2P-RPL. In AODV-RPL, a temporary paired instance is
created between source and destination nodes, in order to transport data
packets between them. The source node which needs to send data initiates an
AODV-RPL instance called RREQ-instance, by multi-casting DIO with the RREQ
message.  The destination node in turn multi-casts or uni-casts a response
message, called the RREP message. The RREP message establishes a path from the
source to destination node, and RREQ message establishes a path from
destination to source.

This paper comprehensively evaluates the performance of RPL and AODV-RPL in a
dense network of constrained nodes. An extensive set of experiments were
conducted, having considered both symmetric and asymmetric links between
multiple node-pairs in the network.

The rest of the paper is organized as follows. Section \ref{Section II} briefly
explains about the pros and cons of traditional RPL and P2P-RPL routing
protocol. Section III briefly overviews of AODV-RPL protocol. Section IV
explains about implementation and experimental results that are being tested in
Cooja simulator with Contiki operating system. Section. V ends up with the
Conclusion and Future work.

\section{Related Work}\label{Section II}

RPL,the IPv6 distance vector routing protocol for LLNs, is designed to
support multiple traffic flows through a root-based Destination-Oriented
Directed Acyclic Graph (DODAG) \cite{RFC6550}. With RPL protocol, the data
packets have to either traverse the root in non-storing mode, or need to
traverse a common ancestor in storing mode. RPL may suffer severe traffic
congestion near the DAG root \cite{RFC6997,RFC6998} and routes may be
sub-optimal.

To discover optimal paths for P2P traffic flows in RPL protocol, a
reactive based P2P-RPL \cite{RFC6997} protocol is proposed which specifies a
temporary DODAG with SourceNode acts as temporary root.P2P-RPL uses address
vector for routing in both stored and non-stored mode which makes it less
efficient in stored mode. The protocol is been implemented and evaluated in
\cite{6064397}.

Both RPL and P2P-RPL specify the use of a single DODAG in networks of symmetric links.

But, application-specific routing requirements that are defined in IETF
ROLL Working Group for constrained networks
\cite{RFC5548,RFC5673,RFC5826,RFC5867} may need routing metrics and constraints
related to bi-directional asymmetric links.  For this, \cite{RFC6549} describes
the bi-directional asymmetric links for traditional RPL with Paired DODAGs, for
which the DAG root (DODAGID) is common for two Instances. With this,
application-specific routing requirements for bidirectional asymmetric links in
base RPL can be satisfied. %But, P2P-RPL for Paired DODAGs require two DAG roots
%namely one for the SourceNode and another for the TargetNode due to temporary
%DODAG formation.
For bi-directional asymmetric links in constrained networks,
AODV-RPL specifies P2P route discovery, utilizing  RPL with a new Mode of
Operation (MoP).  For asymmetrical links, AODV-RPL has two multicast messages,
one from SourceNode to TargetNode and another from TargetNode to SourceNode.
With AODV-RPL,need of address vector is completely eliminated for symmetrical
links which can significantly reduce the control packet size that is important
for constrained LLN networks.

This paper compare RPL with AODV-RPL for both symmetric and asymmetric
links with PDR, delay and number of hops using LLNs border router.

\section{Overview of AODV-RPL}\label{Section.III} In AODV-RPL, reactive
route discovery is initiated by creating a temporary DAG rooted at the
SourceNode. Paired DODAGs (RREQ-Instance and RREP-Instance) are constructed
according to AODV-RPL Mode of Operation (MoP) during route formation between
the SourceNode and TargetNode. The RREQ-Instance is formed by multi-casting a
route control messages from SourceNode to TargetNode whereas the RREP-Instance
is formed by multi-casting (asymmetric)/uni-casting(symmetric) route control
messages from TargetNode to OrigNode. Intermediate routers in LLN changes join
the Paired DODAGs based on the rank as calculated from the DIO message.

An LLN node originating the AODV-RPL message supplies the following
information in the DIO header:
\begin{itemize}

\item ’S’ bit       : Symmetric bit added in the traditional
DIO object.
\item MOP           : Mode of Operation in the DIO object MUST be set to ”5”
for AODV-RPL DIO	message.
\item RPLInstanceID : Instance-ID of paired DODAGs of AODV-RPL Instance.
\item DODAGID       : IPv6 address of the device that initiates the AODV RREQ-Instance
\item Rank          : Node rank corresponding  to AODV-RPL Instance.
\end{itemize}


The ’S’ bit in DIO base object is set to '1' mean that the route from
SourceNode to TargetNode is symmetric. When the RREQ-Instance arrives over an
interface that is known to be symmetric, and the ’S’ bit is set to 1,then it
remains set at 1. When the 	RREQ-Instance arrives at any of the
intermediate node that 	is not known to be symmetric, or is known to be
asymmetric 	then the ’S’ bit in the DIO base object is set to be 0. The
Expected Transmission Count (ETX) value is used to find out whether the link at
the intermediate node is symmetric 	(S=1) or asymmetric (S=0). When the ’S’
bit arrives with already set to be ’0’ then it is set to be ’0’ on
re-transmission at the intermediate nodes. Based on the ’S’ bit
received in DIO-RREQ-Instance, the TargetNode decides whether or not the route
is symmetric before transmitting the RREP-Instance message upstream towards the
SourceNode.

The RREQ-Instance is used for application data transmission from TargetNode to
SourceNode and RREP-Instance is used for application data transmission from
SourceNode to TargetNode.

\section{Implementation}\label{Section.IV}
In order to study the behaviour of RPL and AODV-RPL, the AODV-RPL specification
has been implemented on Contiki \cite{contiki}, an operating system for
wireless sensor networks used and actively developed by a wide industrial and
academic community. Contiki was chosen because it includes an IPv6 stack with
6LoWPAN support, as well as it had RPL implementation, which was used as
basis for  AODV-RPL implementation and simulations can be also be conducted
using Cooja simulator.

\subsection{Experiments}
The experiments consisted 16 node and 38 node networks as shown in figures
\ref{fig:topology_1} and \ref{fig:Topology_2}. Node-1 is assumed to be a LLN
border where LLN traffic is re-routed through traditional wired backbone
network.

The simulations were done in Cooja \cite{contiki}, an emulator for
wireless sensor networks. The experimental setup used are
tabulated in table \ref{tab1:simsetup}
\begin{table}[h!]
\centering
\begin{tabular}{llllllll}
\toprule
\midrule Platform & MSP430 based motes \\ MAC & CSMA  \\ Network stack
& IPv6/6LoWPAN \\ Radio duty cycle (RDC) & null RDC \\  Application & UDP
server/ UDP client\\  Application data length & 40 Bytes \\ Application data
rate & Fixed \\  MAC max retries & 3\\ \bottomrule \\ \end{tabular}
\caption{Used COOJA Simulation configuration } \label{tab1:simsetup}
\end{table}

Radio medium used in Contiki is Directed Graph Radio Medium (DGRM) because it
is easier to configure link states between nodes like RSSI, PRR and LQI so on.
Random SourceNode and TargetNode pairs were considered and experiments are
conducted  making  links symmetric and asymmetric between SourceNode and
TargetNode. Links were	made asymmetric between random SourceNode and
TargetNode pairs by varying Packet Reception Ratio (PRR) values provided in
DGRM configuration. With AODV-RPL, a route discovery to the TargetNode was
initiated by the  SourceNode node and route reply initiated by TargetNode to
SourceNode was recorded. For RPL, a global DAG uses DIO to establish upward
routes and the DAO mechanism in non-storing mode in order to provide paths from
SourceNode to TargetNode as described in Section II. The number of hops taken
to reach TargetNode,delay experienced and PDR achieved were recorded.  Such
experiments were repeated for different  SourceNode and TargetNode pairs with
varying link characteristics.

\begin{figure}[h!]
\includegraphics[width=0.5\textwidth]{./topology_1}
\caption{Topology with 16 nodes}
\label{fig:topology_1}
\end{figure}
\begin{figure}[h!]
\includegraphics[width=0.5\textwidth]{./Topology_2}
\caption{Topology with 38 nodes}
\label{fig:Topology_2}
\end{figure}

\subsection{Experimental Results and Observation}

Experiments with arbitrarily selected SourceNode and TargetNode pairs were
conducted, and we recorded the data pertaining to the number of hops, PDR and
the delay incurred. We categorized the total number of routes as a measure of
the number of hops taken by both the protocols between multiple node-pairs. The
table \ref{tab:hops-usage} shows the percentage of routes that AODV-RPL and RPL
use, and the number of hops taken by each route. We present the individual
performance of both the protocols below

%\subsubsection{AODV-RPL performance }

\begin{table}[h!]
\centering
\begin{tabular}{lllllll}
\toprule
\emph{Number of hops } & \emph{Percentage }  & \emph{ Percentage }\\
\emph{used} & \emph{of AODV-RPL routes} & \emph{of RPL routes}\\
\midrule 2 &  5\% & -\\ 3 & 25\% & 7\% & \\ 4 & 40\% & 30\% & \\ 5 & 15 & 13\%& \\6 & 15\% & 38\% &\\7 &-&10\%\\
\bottomrule \\
\end{tabular}
\caption{Number of hops used by percentage of route by RPL and AODV-RPL} \label{tab:hops-usage}
\end{table}

It can be inferred from the table \ref{tab:hops-usage} that on an average, AODV-RPL
takes far less number of hops than RPL. About 40\% of AODV-RPL routes use 4 hops,
whereas, about 38\% of routes in RPL use 6 hops.

%\begin{figure}[h!]
%\includegraphics[width=0.5\textwidth]{./summaryRoute}
%\caption{Summary route comparison of RPL  and AODV-RPL}
%\label{fig:route-length}
%\end{figure}
%As we can observe from figure \ref{fig:route-length}, 50\% packets in AODV-RPL
%takes 3 to 4 hops and  45\% packets takes maximum of 3 hops, where as in RPL
%30\% of packets takes 3 to 4 hops and 45\% of packets takes 6 to 7 hops.

\begin{figure}[h!] \includegraphics[width=0.5\textwidth]{./summaryDealy}
\caption{Summary: delay of RPL and AODV-RPL} \label{fig:delay} \end{figure}

Another metric considered for comparison is the delay experienced by the
packets traversing the network. We present the histogram of delay experienced
by these packets as shown in the plot \ref{fig:delay}. The percentage of
packets sent are plotted along the y-axis and the delay(ms) is along the
x-axis. We observed from figure \ref{fig:delay}, that 45\% of packets in
AODV-RPL experience (40-50)ms delay and 29\% of the packets experienced
(52-60)ms delay. Whereas in RPL, 37\% of packets experienced delay between
(55-65)ms and 25\% of packets experienced delay between (70-80) ms. Therefore,
we can deduce from the data that AODV-RPL takes less delay compared to RPL.

One of the reasons for the increased delay in RPL is due to the fact that the
packet must travel all the way to a common ancestor, or sometimes to the DODAG
root itself. However, AODV-RPL on the other hand finds the shortest path
on-demand. As the number of hops increase, the cumulative delay due to
re-transmissions between every node-pair also increase. We can also observe that
PDR is less in RPL compared to AODV-RPL, as packets are lost due to congestion
at border-router.

Experimental results of a single SourceNode and TargetNode pair has been
considered below. The histograms shown in figure \ref{fig:delayS10D13-1} and
figure \ref{fig:delayS13D10_1} demonstrate the delay experienced when
SourceNode with ID 13 sent packets to TargetNode with ID 10 and subsequent
reply from Node-ID 10 to Node-ID 13 respectively. It can be observed that the
reply packets take a different path than the sender packet. This is due to the
asymmetric nature of the links.

\begin{figure}[h!] \includegraphics[width=0.5\textwidth]{./delayS13D10_1}
\caption{Packet delay: SourceNode 13 and TargetNode  10}
\label{fig:delayS13D10_1} \end{figure}

\begin{figure}[h!]
\includegraphics[width=0.5\textwidth]
{./delayS10D13-1} \caption{Packet
delay:  SourceNode 10 and TargetNode 13} \label{fig:delayS10D13-1} \end{figure}

From the above graph, it can be deduced that the delay experienced by RPL is
higher than the delay experienced by AODV-RPL. Another observation is that the
path in RPL changes over time. Initially, DODAG establishes the upward routes,
after an exchange of few packets between a node-pair, RPL tries to find a
better path. This causes a change in the parent and subsequently, a change in
the path, and leads to packet loss. However, in AODV-RPL a direct path between
the node-pairs will be established and the same path is retained throughout the
instance. The change in delay due to symmetric and asymmetric links for the
same node-pairs has also been observed.

We also noted that the time between originating the route request and receiving
the route reply was measured to be 550 milliseconds on average, as trickle
timer is not used for the first RREQ/RREP message, node thet received
RREQ/RREP, immediately uni-cast/multi-cast RREQ/RREP. For RPL as in figure
\ref{fig:Topology_2} the route taken by data packets from  SourceNode 30 to
target node 7 is to \verb|30->25->22->17->12->13->10->1->2->7| and reply packet
from node 7 to node 30 is \verb| 7->2->1->10>13->17->21->30|, ie  in case of
links being asymmetric, request and  packets take different routes and number
of hops are also different. In case of AODV-RPL, the path taken for request
packet is \verb|30->21->16->7|  and relay packet takes\verb|7->16->21->30|.

\begin{figure}[h!]
\includegraphics[width=0.5\textwidth]{./brused}
\caption{Percentage of routes using RPL border router}
\label{fig:brused}
\end{figure}
In the overall simulation we can observe as in \ref{fig:brused} that the
percentage of routes pass through RPL-Border-router in case of AODV-RPL is
significantly minimum RPL, which reduces congestion at  RPL-Border-router.


\begin{table}[h!]
\centering
\begin{tabular}{lllllll}
\toprule
\emph{Hop difference } & \emph{Percentage }  \\
\emph{ } & \emph{of routes experienced} \\
\midrule less than 2 &  42\% & \\ 2 or more & 58\%  \\
\bottomrule \\
\end{tabular}
\caption{Number of hops used by percentage of route by RPL and AODV-RPL} \label{tab:hops-difd}
\end{table}


We can observe from  graph \ref{tab:hops-difd} that, 42\% the AODV-RPL and RPL routes  experience route length difference of  0 or 1 hops. Investigation provides the information that if the shortest hop distance between the node pairs is more,then even AODV-RPL can take route length as equal to RPL route length.
Consider for example in figure \ref{fig:topology_1} node pairs (10-16)
experience 6 as shortest hop length, which is route length for both AODV-RPL
and RPL. The route taken by RPL is (10-5-2-1-4-9-16) and for AODV-RPL is(10-11-12-13-14-15-16), and in topology \ref{fig:Topology_2} between node pair (35-33) AODV-RPL uses (35-36-32-37-33)  and RPL uses (35-36-1-37-33),  even though RPL uses border router, when network is not dense it
And also when TargetNode is one of its parent towards
DODAG root, then route length difference RPL and AODV-RPL are less than 2 hops.
These situations AODV-RPL can be avoided,so that can reduce AODV-RPL control
packet overhead.

From these we can conclude an hybrid approach of choosing between RPL and
AODV-RPL, depend application requirement  is recommended.

A few points to be noted here is that, the draft \cite{I-D.ietf-roll-aodv-rpl}
doesn't specify behaviour of packets prior to the establishment of routes by
AODV-RPL. The first AODV-RPL DIO received is used for path establishment, and
the procedure for subsequent DIO is not addressed. The life time of the
ADOV-RPL established routes are also not specified.


\section{Conclusion} \label{Section.V}

Through extensive simulatons on Cooja simulator we evaluated the performance of
AODV-RPL, a P2P routing protocol draft standard that is adopted by IETF's roll
working group. The message reliability and delay performance comparison between
stored mode of traditional RPL operation and AODV-RPL for P2P flows between
various node-pair combinations show that AODV-RPL performs better under
different network toplogies, node densities, and links exhibiting symmetric and
asymmetric properties. While this result is not surprising, we note that there
are P2P routes where the hop distance difference between routes generated by
traditional RPL and AODV-RPL is 2 and less. This crucial observation leads us to
conclude that P2P flows between certain node pairs can reuse the routes already
made available by traditional RPL instead of triggering reactive AODV-RPL
routing mechanism with the associated routing protocol overhead, if the relative
benefits offered by AODV-RPL is not significant. A simple hybrid routing
strategy that uses combination of traditional RPL and AODV-RPL routes is
proposed in the paper.

\section{Acknowledgement}

This work was supported  by the Ministry of Electronics and Information
Technology (MeitY),India. We thank the authors of AODV-RPL draft for their
inputs in implementation

\bibliographystyle{unsrt}
\bibliography{references}
\end{document}

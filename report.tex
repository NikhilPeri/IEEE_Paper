\documentclass[conference, letterpaper]{IEEEtran}
%\documentclass[a4paper]{article}

\ifCLASSINFOpdf \else \fi

\usepackage[pass]{geometry} \usepackage{tabularx} \usepackage{graphicx}
\usepackage{amssymb} \usepackage{algorithm} \usepackage{algorithmic}
\usepackage{amsmath} \usepackage{booktabs} \usepackage{caption}

\usepackage{cite} \usepackage[numbers,sort&compress]{natbib}
\captionsetup[table]{position=top,labelfont={sc},textfont={sl}}

\renewcommand{\thetable}{\Roman {table}}

\title{ Use of Model Predictive Control (MPC) for Rocket Altitude Correction }

\author{ \IEEEauthorblockN{Nikhil Peri, Anthony Lin, Manit Ginoya, Paul Buzuloiu } \
\IEEEauthorblockA{ECE Department, Indian Institute of Science \\ \{nperi104, alin102
mgino015, pbuzu025\}@uottawa.ca} }

\begin{document}
\maketitle
\begin{abstract}
\end{abstract}

Keywords—Model Predictive Control, Aerospace.  \section{Introduction}

Rockets unlike other aircraft have high speed and dynamic flights, as a result
rocket control systems have to be extremely responsive and precise. Classical
control systems based on observed sensor feedback would not be able to meet the
demands of rocket flight since the latency between plant actuation affecting the
the physical world and detecting that change through sensor observations is too
slow for such dynamic flight enviroments. Model Predictive Control (MPC) solves
these problems by introducing state estimation.  This process involves maintianing
a kenetic


\section{Implementation}\label{Section.IV}

\subsection{Experiments}

\subsection{Experimental Results and Observation}

\section{Conclusion} \label{Section.V}

\section{Acknowledgement}

\bibliographystyle{unsrt}
\bibliography{references}
\end{document}
